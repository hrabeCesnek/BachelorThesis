% -----------------------------------------------
% Vlastní text práce (kapitoly práce)
% -----------------------------------------------

% -----------------------------------------------
\chapter{Calibration of astroparticle detectors}
% -----------------------------------------------
Calibration is a process, which we perform to obtain relationships between measured values by tested device and values given by ethalon. These relationships may be specified by calibration constants, functions or by other mathematical relations. The tested devices can be also calibrated relatively to each other.
\par
%In case of astroparicle detection, we mostly need to ,  , 
\section{Absolute calibration}
% -----------------------------------------------
The term absolute calibration can be specified as a measurement process, which results in obtaining reference between the detector's responsivity and the defined value of some physical quantity induced by an external source. In case of detectors consisting of PMTs or PMT based cameras it is reference between optical power seen by detector and its output photocurrent. Due to the PMT's linearity principle we can define this reference by one constant for every PMT or pixel. However, in real applications, other corrections must be made.

\par
Good example of absolute calibration are large-aperture light drums, which are used at Pierre Auger for fluorescence telescopes calibration.

\par
In many cases the absolute calibration is a complicated process, and thus it is done less frequently than the relative calibration.

% -----------------------------------------------
\section{Relative calibration}



%--------------------------------------------
\section{FAST calibration}
%-----------------------------------------------------
FAST telescope calibration techniques are yet under development. It is necessary to calibrate PMTs both in absolute and relative way, but also the entire telescope as optoelectronical system with mirrors and PMTs. 
% ----------------------------------------------------
\subsection{Flasher}
As was mentioned before, the FAST telescope is equipped with UV LED flasher, which is used to generate light pulses. Calibration by flasher is performed during every shift - two times at the beginning and two times at the end. One time with opened shutter and one time with closed shutter. 
\subsection{YAP pulser}



\subsection{Homogenous light source}
There is 

%------------------------------------------------


% -----------------------------------------------



% -----------------------------------------------
% %%%%%%%%%%%%%%%%%%%%%%%% End of file %%%%%%%%%%%%%%%%%%%%%%%%
