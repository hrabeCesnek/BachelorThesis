% -----------------------------------------------
% Vlastní text práce (kapitoly práce)
% -----------------------------------------------

% -----------------------------------------------
\chapter{Calibration of astroparticle detectors}
% -----------------------------------------------
Calibration is a process, which we perform to obtain relationships between measured values by tested device and values given by used ethalon. These relationships may be specified by calibration constants, functions or by other mathematical relations. The tested devices could be also calibrated relatively to each other.
\par
In case of astroparicle detection, we mostly need to ,  , 
\section{Absolute calibration}
% -----------------------------------------------


% -----------------------------------------------
\section{Relative calibration}
%--------------------------------------------
\section{FAST calibration}
%-----------------------------------------------------
FAST telescope calibration techniques are yet under development. It is necessary to calibrate PMTs both in absolute and relative way, but also the entire telescope as optoelectronical system with mirrors and PMTs. 
% ----------------------------------------------------
\subsection{Flasher}
As was mentioned before, the FAST telescope is equipped with UV LED flasher, which is used to generate light pulses. Calibration by flasher is performed during every shift - two times at the beginning and two times at the end. One time with opened shutter and one time with closed shutter. 
\subsection{YAP pulser}



\subsection{Homogenous light source}
There is 

%------------------------------------------------


% -----------------------------------------------



% -----------------------------------------------
% %%%%%%%%%%%%%%%%%%%%%%%% End of file %%%%%%%%%%%%%%%%%%%%%%%%
