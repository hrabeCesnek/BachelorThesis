% -----------------------------------------------
% Vlastní text práce (kapitoly práce)
% -----------------------------------------------

% -----------------------------------------------
\chapter{FAST telescope}
% -----------------------------------------------
The  Fluorescence  detector  Array  of  Single-pixel  Telescopes  (FAST) is an international project of fluorescence telescope sensitive to UHERCs. 
\par
Until today there are four prototypes in active service. Three of them are situated in Black Rock Mesa site of the Telescope Array experiment in central Utah and one in Argentina near Pierre Auger Observatory.
\par
The main goal of FAST project is to develop a cheap fluorescence telescope, which could be used in future to cover the wide surface area. This new oncoming fluorescence telescope array should be able  
to fully reconstruct the geometry of UHERCs induced UV shover by combining the information from telescopes, which has encountered some event at the same time. 
% -----------------------------------------------
\section{Principle of operation}
% -----------------------------------------------
Main detection part of telescope consists of superreflective UV mirrors and photomultipliers. 


\par
The entire telescope along with monitoring systems and other instruments is situated in a hut with remote shutter, where it is protected from negative metrological phenomena, such as rain or fast wind, but also from dust and aerosols. Exposure of mirrors to any of this phenomena could lead to reduction of theirs reflectivity. It is also neccessary to monitor and protect PMTs from unwanted light sources. Even a low-intensity sources could decrease PMT's service life.
% -----------------------------------------------

\section{Remote control and monitoring}
% -----------------------------------------------

%------------------------------------------------

\section{}
% -----------------------------------------------



% -----------------------------------------------
% %%%%%%%%%%%%%%%%%%%%%%%% End of file %%%%%%%%%%%%%%%%%%%%%%%%
