% -----------------------------------------------
% Vlastní text práce (kapitoly práce)
% -----------------------------------------------

% -----------------------------------------------
\chapter{FAST telescope}
% -----------------------------------------------
The  Fluorescence  detector  Array  of  Single-pixel  Telescopes  (FAST) is an international project of fluorescence telescope sensitive to UHERCs. 
\par
Until today there are four prototypes in active service. Three of them are situated in Black Rock Mesa site of the Telescope Array experiment in central Utah and one in Argentina near Pierre Auger Observatory.
\par
The main goal of FAST project is to develop a cheap fluorescence telescope, which could be used in future to cover the wide surface area. This new oncoming fluorescence telescope array should be able  
to fully reconstruct the geometry of UHERCs induced UV shover by combining the information from telescopes, which has encountered an event at the same time. 

\begin{figure}[H]
 \centering
 \includegraphics[scale = 0.5]{./pictures/fastTheoretical}
 \caption{FAST telescope's design \cite{MALACARI2020102430}.}
 \label{FASThut}
 
\end{figure}


\begin{figure}[H]
 \centering
 \includegraphics[scale = 0.2]{./pictures/FASTReal}
 \caption{FAST telescope \cite{Project}.}
 \label{FASThut}
 
\end{figure}
% -----------------------------------------------
\section{FAST detection and operation scheme}
% -----------------------------------------------
Main detection part of telescope consists of superreflective UV mirrors and photomultipliers. 


\par

\section{Protection hut}
The entire telescope along with monitoring systems and other instruments is situated in a hut with remote shutter, where it is protected from negative metrological phenomena, such as rain or fast wind, but also from dust and aerosols. Exposure of mirrors to any of this phenomena could lead to reduction of theirs reflectivity. It is also neccessary to monitor and protect PMTs from unwanted light sources. Even a low-intensity sources could decrease PMT's service life.
% -----------------------------------------------

\section{Remote control and monitoring}
% -----------------------------------------------
In case of Argentina prototype, the telescope's systems are connected to Pierre Auger network and through it, the telescope could be controlled and monitored over the internet. 

\par
Testing measurements, sometimes refered as shifts, are performed in the night. Their main purpose is to acquire data, which could be later analyzed and compared with data from Los Leones part of Pierre Auger, which has the similar FoV as Argentine FAST. 
The testing measurements require an operator to look after the telescope systems. An operator's duty is to power on and off the DAQ, PMTs' voltage sources, perform calibration, open and close the shutter and mainly check for failures and negative phenomenas. To do that, operator could access webcam (\ref{FASTCam}), meteostation with thermometers, humidity, wind and light sensors and the Allsky camera.

\begin{figure}[H]
 \centering
 \includegraphics[scale = 0.5]{./pictures/operatinFast}
 \caption{FAST telescope with opened shutter from webcam.}
 \label{FASTCam}
 
\end{figure}

%\begin{figure}[H]
% \centering
% \includegraphics{up_logo_bw}
% \caption{View from Allsky camera installed atop the hut. It gives the information %of sky quality by comparing visible stars with theoreretical star map.}
% \label{AllskyCam}
 
%\end{figure}







% -----------------------------------------------
% %%%%%%%%%%%%%%%%%%%%%%%% End of file %%%%%%%%%%%%%%%%%%%%%%%%
