% -----------------------------------------------
% Vlastní text práce (kapitoly práce)
% -----------------------------------------------

% -----------------------------------------------
\chapter{Astroparticle detection}
% -----------------------------------------------
More than 100 years have passed since Victor Franz Hess first encontered cosmic radiation. Since those times the techniques and methods of detection have been strongly improved. We have moved up from elevating electroscopes by ballons to observe growing electric charge to specialized techniques, which allows us to measure particles' energies, trajectories, etc.

% -----------------------------------------------
\section{Cosmic rays and particles}
% -----------------------------------------------
Cosmic rays is a term for radiation and energetic particles striking earth atmosphere with an origin in a space sources (neutron stars, supernovas, black holes, etc). We divide them into two major groups - primary and secondary. Primary cosmic rays are the original cosmic particles, which strike the Earth's athmosphere. Secondary cosmic rays (also refered as showers) are particles, which have origin in particle interaction between primary cosmic rays and the athmosphere.
\par
\subsection{Primary cosmic rays}
Primary cosmic rays consist of protons ($95 \%$), helium nuclei ($4 \%$), electrons and other heavy nuclei (up to iron). However, only the energetic rays make their way to the athmosphere. The Earth's magnetic field affects their trajectories and prevents the low-energetic (less than 100 MeV) particles from arriving to the athmosphere \cite{Kliewer}. 
\par
Part of primary cosmic rays are also Ultra-high energy cosmic rays (UHECRs), which we refer in the next chapter.
\par 
Neutrinos are also a part of cosmic radiation, but their interaction with matter is very rare, so they are very hard to detect. The special underwater detectors are developed to detect some of them. 
\subsection{Secondary cosmic rays}
Secondary cosmic rays are created by interaction of high-energetic particles of primary component with air's nucleis, such as nitrogen. They consist of low-energetic and high-energetic muons, gamma photons, electrons and positrons. Most of muons travel up to the earth's surface although their half-life is only about 2.2 microseconds before they decay into electrons. Due to their high relativistic speeds, their half-life is increased for external observers. 




% -----------------------------------------------
\section{Ultra-high energy cosmic rays (UHECRs)}
UHECRs are particles with energies from $10^{18}$ to $10^{20}$ eV, which is much more than particles created on Cern's Large hadron colider (LHC) with energies about $10^{13}$. Due to their high energies, the trajectory remains nearly unchanged by space magnetic fields \cite{Benjamin_Skuse}.
\par
UHECRs' origin is yet unknown, but it is supposed and experimentally proved that they come from outside of the Milky Way. Some theoretical physicists expects, that the one possible source of UHECRs acceleration are the starburst galaxies. One of the UHECR possible candidate is proton. 
\par
In many particle physics experiments, some form of calorimeter is used to determine the particles' energy and direction. In case of UHECRs, the air molecules of athmosphere have the function of calorimeter.
When the UHECR interacts with athmospheric nuclei, the cascade of particles is induced. This cascade concists of three main components: electromagnetic, hadronic and muonic. It is also refered as extensive air shower (EAS).
\par
In case of ultra-high energy proton striking an air molecule, kaons, baryons, nuclear fragments and mostly pions are created. Together, they are refered as hadronic component. The pions with short lifetime decay into electromagnetic sub-shower. The behaviour of pions with longer lifetime is energy-dependent. At higher energies they reinteract with atmospheric nuclei and feed hadronic and electromagnetic component. At lower energies they decay into muonic component with muonic neutrinos. The muons with short lifetime decay into electrons and positrons with neutrinos, which joins the electromagnetic cascade, while the others reach the Earth's surface carrying the energy, which we are unable to detect. The UHECR's cascade scheme could be seen on picture \ref{cascade}. 
\begin{figure}[H]
 \centering
 \includegraphics{up_logo_bw}
 \caption{UHECR's cascade, taken from nevimco \ref{nevimco}.}
 \label{cascade}
 
\end{figure}

\par

The electromagnetic cascade consisting of electron, positrons, gama photons carries the most of UHECR's original energy. With increasing depth, the cascade develops by decaying the photons into electron and positron pairs, which again emit photons and this process repeats. The cascade develops until the the energy treshold is reached, where the ionization is higher than the radiation losses. \ref{Tomankova2016_1000061954}.


\par
One of the main parameters of electromagnetic cascade is $X_{max}$ - the point along the shower axis, where is the maximum of deposited energy.

% -----------------------------------------------
\section{UHECRs detection techniques}
Nowadays there are three main proven techniques to detect UHERCs - 
air fluorescence, surface particle detection and Cherenkov light detection.

\subsection{Air fluorescence}
The electrons and positrons from electromagnetic cascade excite nitrogen molecules, which then deexcite and emit fluorescence light in UV spectre with two main wavelengths 337 and 357 nm. The intensity of this light is directly proportional to the calorimetric energy of the shower.
\par
However, the intensity of this light is very low, and thus the measurements must be done in dark nights with no light smog. Even a low intensity of external sources could led to detection of false events or worse - the unwanted light could damage the very sensitive equipment. 
\par
The detection equipment are mostly the superreflective UV mirrors which focus part of the UV shower into the PMT or into the camera based on PMTs. It it necessary to have atmospheric monitoring system (temperature, humidity, pressure etc.) along with the detection part, because the fluorescence is highly dependent on these conditions.  
\par
The main advantage of fluorescence detection is the fact, that by measuring and integrating the light profile we are able to calculate the total calorimetric energy. However, corrections   

\par
The FAST telescope, on which we focus in this thesis, is considered to be a fluorescence telescope.
\subsection{Surface particle detection}
Surface particle detection is a method, which is based on detecting individual particles from EAS. There are two proven surface techniques to detect EAS - by scintillators or water Cherenkov detectors. To detect the EAS full geometric distribution, the huge surface needs to be covered by surface detectors. 


\par
Compared to fluorescence detection method, the surface detectors don't require the dark night and any light exposure is not a threat for them.  



\subsection{Cherenkov light detection}


\subsection{Hybrid detection}

\section{UHECRs observatories}
\subsection{Pierre Auger observatory}
% -----------------------------------------------

\subsection{Telescope array project}
% -----------------------------------------------
% %%%%%%%%%%%%%%%%%%%%%%%% End of file %%%%%%%%%%%%%%%%%%%%%%%%
