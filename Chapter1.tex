% -----------------------------------------------
% Vlastní text práce (kapitoly práce)
% -----------------------------------------------

% -----------------------------------------------
\chapter{Astroparticle detection}
% -----------------------------------------------
More than 100 years have passed since Victor Franz Hess first encontered cosmic radiation. Since those times the techniques and methods of detection have been strongly improved. We have moved up from elevating electroscopes by ballons to observe growing electric charge to specialized techniques, which allows us to measure particles' energies, trajectories, etc.

% -----------------------------------------------
\section{Cosmic rays and particles}
% -----------------------------------------------
Cosmic rays is a term for radiation and energetic particles striking earth atmosphere with an origin in a space sources (neutron stars, supernovas, black holes, etc). We divide them into two major groups - primary and secondary. Primary cosmic rays are the original cosmic particles, which strike the Earth's athmosphere. Secondary cosmic rays (also refered as showers) are particles, which have origin in particle interaction between primary cosmic rays and the athmosphere.
\par
\subsection{Primary cosmic rays}
Primary cosmic rays consist of protons ($95 \%$), helium nuclei ($4 \%$), electrons and other heavy nuclei (up to iron). However, only the energetic rays make their way to the athmosphere. The Earth's magnetic field affects their trajectories and prevents the low-energetic (less than 100 MeV) particles from arriving to the athmosphere.
\par
Part of primary cosmic rays are also Ultra-high energy cosmic rays (UHECRs), which we refer in the next chapter.
\par 
Neutrinos are also a part of cosmic radiation, but their interaction with matter is very rare, so they are very hard to detect. The special underwater detectors are developed to detect some of them. 
\subsection{Secondary cosmic rays}
Secondary cosmic rays are created by interaction of high-energetic particles of primary component with air's nucleis, such as nitrogen. They consist of low-energetic and high-energetic muons, gamma photons, electrons and positrons. Most of muons travel up to the earth's surface although their half-life is only about 2.2 microseconds before they decay into electrons. Due to their high relativistic speeds, their half-life is increased for external observers. 




% -----------------------------------------------
\section{Ultra-high energy cosmic rays (UHECRs)}
UHECRs are particles with energies from $10^18$ to $10^20$ eV, which is much more than particles created on Cern's Large hadron colider (LHC) with energies about $10^13$. Due to their high energies, the trajectory remains nearly unchanged by space magnetic fields.
\par
UHECRs' origin is yet unknown, but it is supposed and experimentally proved that they come from outside of the Milky Way. Some theoretical physicists expects, that the one possible source of UHECRs acceleration are the starburst galaxies.
\par
When the UHECR interacts with athmospheric nuclei,

% -----------------------------------------------
\section{UHECRs detection techniques}


\subsection{Pierre Auger observatory}
% -----------------------------------------------

\subsection{Telescope array project}
% -----------------------------------------------
% %%%%%%%%%%%%%%%%%%%%%%%% End of file %%%%%%%%%%%%%%%%%%%%%%%%
