% -----------------------------------------------
% Vlastní text práce (kapitoly práce)
% -----------------------------------------------

% -----------------------------------------------
\chapter{Instrumentalization and measurement preparation}
% -----------------------------------------------
To perform all of neccesary measurements we need to use various types of optical and electronical equipment.
% -----------------------------------------------
\section{Integration sphere}
% -----------------------------------------------
The Integration sphere (IS) is a special optical equipment, which can be used either as extended uniform light source (EULS) or with spectrometer in determining the material reflectance. In our experiments we use general purpose Labsphere (Fig. \ref{Labsphere}).

\begin{figure}[H]
 \centering
 \includegraphics{up_logo_bw}
 \caption{General purpose Labsphere.}
 \label{Labsphere}
 
\end{figure}
\par
The IS inner surface consist of white optical diffusive material (BaSO$_4$ and Polytetrafluoroethylene). The IS also contains several circular apertures, which are called input/output ports. They can be used to mount detectors or optical sources or left free to let light flux enter or exit IS. 
\par
The inner surface is part where light intergration happens. The effect which takes place here is known as Lambertian scattering. After one spot of inner surface is hit by a ray, the energy should be uniformly radialy distributed. In output port this produces a homogenous light source. The homogenity decreases with increasing number and sizes of input/output ports.
\par
Using optical source with IS requires baffle to prevent source's light flux or its part to exit IS without integration.
\par
More deeper explanation of IS working principles and characterization of optical properties of identical IS, which we use, can be found in \cite{VACULA2021167169}.
\par
For our pusrposes, in case of FAST calibration, we use IS as EULS in UV spectre. In case of testing optical calibration source, we don't even care about homegenity. The reason why we use IS in this case is that it focuses the entire optical power of the source into output ports, where  our detectors are mounted, and blocks any other external light source, which could affect our detectors.
% -----------------------------------------------

\section{Photomultiplier tube}
% -----------------------------------------------
Photomultiplier tube (PMT) is considered to be a high voltage optoelectronical part. It allows us to measure very low intesity optical signals. PMT is also characterized by high amplification, low noise and stability. 
\subsection{Operating principle}
PMT consists of 6 main elements, which can be seen on scheme \ref{PMT scheme}.

\begin{figure}[H]
 \centering
 \includegraphics{up_logo_bw}
 \caption{Photomultiplier tube scheme.}
 \label{PMT scheme}
\end{figure}


\par
The input photon with sufficient energy, which strikes the PMT's photocathode, excites photocathode's electron. This electron then follows electrostatitc field to the first dynode of the electron multiplier, where it induces secondary emission of more electrons. These electrons are then attracted by the next dynode, where the emission process repeats. After few times of multiplying electron number over dynodes, the electrons are then collected by 
the anode, which is situated on the end of the electron multiplier. The anode output current is then converted to voltage signal by appropriate load resistor or by operational amplifier current-to-voltage circuit.
\par
As all other laboratory instruments, which are based on accelerating electrons, such as electron microscopes, the photomultiplier's main parts must be kept in vacuum. To maintain vacuum, the photomultiplier is surrounded by special glass envelope. To avoid mechanical damage of the glass envelope, the entire photomultiplier is situated in a plastic tube.
\par
One of the basic adjustable characteristic of PMT is its gain. The gain is defined as:

\begin{equation}
G = \frac{I_\textrm{a}}{I_\textrm{p}},
\end{equation}
where $I_\textrm{a}$ is the anode current and $I_\textrm{p}$ is the input photocurrent from the photocathode.
\par
In case of ideal, noiseless PMT, we can adjust gain by variing the supply voltage. By variing supply voltage we can adjust gain according to an euqation:

\begin{equation}
\frac{G_2}{G_1} = (\frac{V_2}{V_1})^{\alpha N},
\end{equation}
where $G_2$ and $G_1$ are gains at supply voltages $V_2$ and $V_1$. $\alpha$ is coefficient given by dynode material and $N$ is the number of dynodes.

Other effects, such as temperature, may also vary PMT's gain, and it is neccesary to keep them on constant value or monitor.



\subsubsection{Window}

The photocathode is superimposed by glass window, whose main purpose is to admit light of certain wavelengths. Glass materials are characterized by the spectral sensitivity to wavelengths. For transparency in UV spectre, it is advised to use borosilicate or fused silica glasses.


\subsubsection{Photocathode}

The photocathode is the only light-sensitive part of PMT. It transfers the light flux into electric current.
\par
One of its main parameters is quantum efficenty. It is refered as ratio of emitted photoelectrons to the number of incident photons expressed as a percentage. It is generally less than 35 \%. For measurement, the more practical parameter is cathode radiant sensitivity. It is the ratio of photocathode current to an incident light power, which is expressed in mA/W.
\par
Photocathode material must be sensitive to certain wavelengths, which we want to detect with the PMT, and must have sufficent quantum efficenty. Prefered materials are usually alkali antimodes.
\subsubsection{Electron multiplier}
The electron multiplier consists of dynodes and one anode.
Dynodes are electrodes, which produce more electrons through secondary emission. To maintain elecrostatic field between dynodes, each of dynodes is held on different potential. This is achieved by using the voltage divider. Every resistor in the divider sets the potential of one diode according to its resistivity.
\par
All of the photoelectrons emitted by photocathode should be ideally collected by the first dynode. However, many of them could be diverted from their path to dynode due to various effects. The parameter, which characterizes this, is the collection efficiency. The Collection efficiency is a probability that photoelectron will strike area of the first dynode. 
\par
There are few types of dynodes arrangements. On the fig. \ref{PMT scheme} is the classic linear-focusing multiplier. 


\subsubsection{Voltage divider and voltage adjustement}
Voltage divider could be a simple resistor serial network, which divide high input voltage between the dynodes. 
\par
It is neccessary to consider, that the multiplier current density increases in direction to the anode, so it tends to lessen the voltage between last dynode and anode. This phenomena can shake the potential levels across the entire multiplier. One way to reduce the impact on PMT's behaviour is to choose the proper resistor values of the divider. 
\par
The resistor values could same for all the dynodes, but for some applications it is better to have progresive voltage distribution, which increases from cathode to anode, or intermediate distribution with highest values on the beginning of the multiplier.

\par
In some aplications, where high anode current peaks are expected, the divider can be filled with reservoir capacitors, which prevent the temporaly charge exhaustion of the dynodes. In pulse mode, the unwanted oscillacions on dynodes may occur, in that case, it is desirable to connect additional damping resistor to the divider.


\subsection{Gain and sensitivity}

\subsection{Noises and Dark current}

\subsection{Operating life and degradation}
The operating life of PMT is defined as the time required for anode sensitivity to be halved.


\subsection{FAST's PMTs}

\subsection{Calibration}

%------------------------------------------------

\section{Silicon PM}
% -----------------------------------------------

%------------------------------------------------

\section{Hardware for experiment control}
% -----------------------------------------------
\subsection{Raspberry Pi}

\subsection{STM32 based microcontrolers}

\section{Sensors and other electronics components}

\subsection{MPU6050}

\subsection{servo motors}

\subsection{Dallas DS18B20 thermometer}
% -----------------------------------------------
% %%%%%%%%%%%%%%%%%%%%%%%% End of file %%%%%%%%%%%%%%%%%%%%%%%%
