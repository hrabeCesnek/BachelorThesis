% -----------------------------------------------
% Vlastní text práce (kapitoly práce)
% -----------------------------------------------

% -----------------------------------------------
\chapter{Instrumentalization and measurement preparation}
% -----------------------------------------------
To perform all of neccesary measurements we need to use various types of optical and electronical equipment.
% -----------------------------------------------
\section{Integration sphere}
% -----------------------------------------------
The Integration sphere (IS) is a special optical equipment, which can be used either as extended uniform light source (EULS) or with spectrometer in determining the material reflectance. In our experiments we use general purpose Labsphere (Fig. \ref{Labsphere}).

\begin{figure}[H]
 \includegraphics{up_logo_bw}
 \caption{General purpose Labsphere.}
 \label{Labsphere}
 
\end{figure}
\par
The IS inner surface consist of white optical diffusive material (BaSO$_4$ and Polytetrafluoroethylene). The IS also contains several circular apertures, which are called input/output ports. They can be used to mount detectors or optical sources or left free to let light flux enter or exit IS. 
\par
The inner surface is part where light intergration happens. The effect which takes place here is known as Lambertian scattering. After one spot of inner surface is hit by a ray, the energy should be uniformly radialy distributed. In output port this produces a homogenous light source. The homogenity decreases with increasing number and sizes of input/output ports.
\par
Using optical source with IS requires baffle to prevent source's light flux or its part to exit IS without integration.
\par
More deeper explanation of IS working principles and characterization of optical properties of identical IS, which we use, can be found in \cite{VACULA2021167169}.
\par
For our pusrposes, in case of FAST calibration, we use IS as EULS in UV spectre. In case of testing optical calibration source, we don't even care about homegenity. The reason why we use IS in this case is that it focuses the entire optical power of the source into output ports, where  our detectors are mounted, and blocks any other external light source, which could affect our detectors.
% -----------------------------------------------

\section{Photomultiplier}
% -----------------------------------------------
Photomultiplier (PMT) is considered to be a high voltage optoelectronical part.
%------------------------------------------------

\section{Silicon PM}
% -----------------------------------------------

%------------------------------------------------

\section{Hardware for experiment control}
% -----------------------------------------------
\subsection{Raspberry Pi}

\subsection{STM32 based microcontrolers}

\section{Sensors and other electronics components}

\subsection{MPU6050}

\subsection{servo motors}

\subsection{Dallas DS18B20 thermometer}
% -----------------------------------------------
% %%%%%%%%%%%%%%%%%%%%%%%% End of file %%%%%%%%%%%%%%%%%%%%%%%%
