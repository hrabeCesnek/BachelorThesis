% -----------------------------------------------
% Vlastní text práce (kapitoly práce)
% -----------------------------------------------

% -----------------------------------------------
\chapter{Instrumentalization and measurement preparation}
% -----------------------------------------------
To perform all of neccesary measurements we need to use various types of optical and electronical equipment.
% -----------------------------------------------
\section{Integration sphere}
% -----------------------------------------------
The Integration sphere (IS) is a special optical equipment, which can be used either as extended uniform light source (EULS) or with spectrometer in determining the material reflectance. In our experiments we use general purpose Labsphere (Fig. \ref{Labsphere}).

\begin{figure}[H]
 \includegraphics{up_logo_bw}
 \caption{General purpose Labsphere.}
 \label{Labsphere}
 
\end{figure}
\par
The IS internal surface consist of optical diffusive material (
The homogenity decreases with increasing number and sizes of input/output ports. More deeper explanation of IS working principles and characterization of optical properties of identical IS, which we use, can be found in \cite{VACULA2021167169}.
\par
For our pusrposes, in case of FAST calibration, we use IS as EULS in UV specre. In case of testing optical calibration source, we don't even care about homegenity. The reason why we use IS in this case is that it focuses the entire optical power of the source into output ports, where  our detectors are mounted, and blocks any other external light source, which could affect our detectors.
% -----------------------------------------------

\section{Photomultiplier}
% -----------------------------------------------

%------------------------------------------------

\section{Silicon PM}
% -----------------------------------------------

%------------------------------------------------

\section{Hardware for experiment and measurement control}
% -----------------------------------------------
\subsection{Raspberry Pi}

\subsection{STM32 based microcontrolers}
% -----------------------------------------------
% %%%%%%%%%%%%%%%%%%%%%%%% End of file %%%%%%%%%%%%%%%%%%%%%%%%
